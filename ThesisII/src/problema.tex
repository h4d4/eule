\label{ch:problema}
\chapter{Descripción del problema}

En Android, por defecto, el desarrollador no cuenta con mecanismos para
definir políticas de confidencialidad e integridad que regulen
el flujo de información de sus aplicaciones. Siendo complejo prevenir fugas de
información del usuario, puesto que, el desarrollador carece de herramientas que
le garanticen la ausencia de flujos indeseados.\newline
Precisamente, una de las principales preocupaciones de seguridad en aplicativos
Android, es la manipulación de información del usuario.
Así lo evidencia un
estudio reciente de seguridad en dispositivos móviles, publicado por
McAfee\cite{McAfeeReport}, este señala  que una importante cantidad de
aplicaciones Android invaden la privacidad del usuario, reuniendo información
detallada de su desplazamiento, acciones en el dispositivo, y su vida personal.
De este modo, 80\% reúnen información de la ubicación, 82\%
hacen seguimiento de alguna acción en el dispositivo , 57\%
registran la forma de uso del celular (mediante Wi-Fi o
mediante la red de telefonía), y 36\% conocen información de
las cuentas de usuario.\newline
Las motivaciones para este tipo de acciones varían acorde al tipo de
información, por ejemplo: monitorear información de ubicación para mostrar
publicidad no solicitada; seguir las acciones sobre el dispositivo, para conocer
qué aplicaciones son rentables de desarrollar, o para ayudar a aplicaciones
maliciosas a evadir defensas; acceder a información de cuentas del usuario con
fines delictivos; obtener información de contactos y calendario
del usuario, buscando modificar los datos; obtener información del celular 
(número, estado, registro de MMS y SMS) para interceptar llamadas y enviar
mensajes sin consentimiento del usuario.\newline
Con o sin autorización de acceso, existen motivaciones suficientes para que un
tercero desee manipular información del usuario.\newline
Adicionalmente, el informe señala que una aplicación invasiva no necesariamente
contiene malware, y que su finalidad no siempre implica fraude; de las
aplicaciones que más vulneran la privacidad del usuario, 35\% contienen
malware.\newline 
Si bien, aplicaciones invasivas no necesariamente implican
malware y/o acciones delictivas, el cuestionamiento de fondo es la forma y
finalidad con que están accediendo la información, es decir, si información
de usuario manipulada por una determinada aplicación, realmente debería ser
accedida por otros aplicativos del dispositivo, aún cuando sean considerados no
maliciosos; y qué garantías puede ofrecer el desarrollador para que tal acceso,
efectivamente sea consentido.\newline 
La falta de control sobre los flujos de información de la aplicación puede
ocasionar fugas de información, generando problemas de seguridad tanto para
quien la implementa como para quien la usa.\newline
Como contramedida a este problema, la API de Android ofrece herramientas de
seguridad basadas en políticas de control de acceso, y el desarrollador puede
implementarlas en su aplicación. Sin embargo, estos mecanismos se centran en
regular el acceso de los usuarios del sistema a determinados recursos, y no en
verificar qué sucede con la información una vez es accedida.\newline 
Para superar tal carencia, diferentes trabajos de investigación han abordado el
problema de fuga de información en aplicaciones Android, tanto desde un enfoque
dinámico como desde un enfoque estático, la literatura existente al
respecto(TaintDroid\cite{TaintDroid}, Flow-Droid\cite{FlowDroid-Thesis},
DidFail\cite{DidFail}, DroidForce\cite{DroidForce}), indica que la mayoría de
propuestas hacen data-flow analysis mediante técnicas de análisis usando
tainting, partiendo del bytecode. Una característica sobresaliente entre estos
trabajos es el modelo de ataque, puesto que, se centran en analizar aplicaciones
de terceros asumiendo que el atacante provee bytecode malicioso. Guiar el
análisis de aplicaciones propias con el fin de verificar políticas de
confidencialidad e integridad, bajo tales propuestas, puede implicar: mayor
dificultad en el código a analizar, incompletitud en el análisis(under-tainting)
y no detección de flujos implícitos. Esto debido a que, aún cuando el
desarrollador conoce la funcionalidad de su propio código, las optimizaciones
realizadas por el compilador pueden adicionar complejidad al
mismo\cite[pag.~43]{SecureProgramming}; el seguimiento de los datos a través del
programa está centrado en datos marcados, datos no marcados quedan fuera del
análisis;   flujos de datos a través de estructuras de control, por ejemplo, las
sentencias if, permiten inferir valores de datos marcados como source, sin
necesidad de generar flujos explícitos entre sources y sinks, los cuales si
pueden ser detectados por las técnicas de análisis tainting.\newline Otra razón
fundamental para no  analizar aplicaciones propias con tales propuestas es que
están diseñadas para detectar flujos de datos indebidos, y no para garantizar el
cumplimiento de políticas de seguridad en una aplicación.\newline 
Los riesgos de seguridad tras el under-tainting de datos, y la ausencia de
garantías en el cumplimiento de determinadas políticas de seguridad, pueden
superarse mediante control de flujo de información, Information Flow
Control(IFC), puesto que, con esta técnica se analiza estáticamente la
aplicación para identificar todos los posibles caminos que podrían tomar sus
flujos de información, garantizando que a tiempo de ejecución, la
aplicación respeta políticas de seguridad.\newline

Finalmente, partiendo del contexto que se plantea, dónde se cuenta con el código
fuente Android, porque es el propio desarrollador quien requiere evaluar
políticas de confidencialidad en su aplicación, para  garantizarle
al usuario que la aplicación las cumple. Resulta apropiado proveer una
herramienta de apoyo al desarrollador, mediante la cual analice el flujo de
información de la aplicación próxima a liberar, y verifique el cumplimiento de
políticas de seguridad.
 














