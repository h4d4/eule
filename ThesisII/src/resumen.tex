\label{ch:resumen}
\chapter*{
\begin{center}
	Resumen
\end{center} }
Breve resumen del trabajo : contexo, problema, solución propuesta, resultados
alcanzados.

La presente investigación plantea aplicar técnicas de análisis basadas en
control de flujo de información, con el fin de verificar la ausencia de fugas de
información en aplicaciones Android. Puesto que, controlar el acceso y uso de la
información, representa una de las principales preocupaciones de seguridad en
dichos aplicativos.\newline 
Un estudio reciente de seguridad en dispositivos móviles, publicado por
McAfee\cite{McAfeeReport}, revela que en el contexto de aplicativos Android:
80\% reúnen información de la ubicación, 82\% hacen seguimiento de alguna acción
en el dispositivo, 57\% registran la forma de uso del celular (mediante Wi-Fi o
mediante la red de telefonía), y 36\% conocen información de las cuentas de
usuario.\newline
Diferentes trabajos de investigación han abordado el problema de pérdida de
información en aplicativos Android, sin embargo, la literatura científica
existente al respecto, permite inferir que la mayoría de trabajos aplican
técnicas para hacer data-flow análisis a partir del bytecode. De modo que, su
finalidad es detectar fugas de información y no, verificar que el aplicativo
respeta determinadas políticas de seguridad. Así, el desarrollador de la
aplicación carece de herramientas de apoyo para verificar si la aplicación que
implementa, cumple con determinadas políticas de seguridad.\newline