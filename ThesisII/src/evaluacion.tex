\label{ch:evaluacion}
\chapter{Evaluación}
% Ventajas y limitaciones de la solución.\newline 
% Si aplica, evaluación de desempeño.  \newline 
% Si aplica, evaluación de usabilidad.  
% Hay otras soluciones similares? \newline 
% Cuáles son las diferencias y las ventajas y desventajas con respecto a esas soluciones.

\section{Consideraciones de evaluación}
No se consideran flujos de información vía interAppComunicación. Por ejemplo,
varias aplicaciones que se comunican entre sí.

\section{Evaluación conjunto de aplicaciones}
Para la evaluación se parte de DroidBech, benchmark integrado por casos de
prueba para aplicaciones Android, cuyos autores son los mismos creadores de
FlowDroid. Se opta por utilizar DroidBench puesto que, en la literatura
científica consultada al respecto, no se encuentran otros benchmarks diseñados
específicamente para evaluar aplicaciones Android.\newline 
De DroidBech se toma un grupo de testcases evaluables frente a la política de
seguridad establecida, este grupo está integrado por 20 testcases.

La tabla \ref{tab:descripApps0} describe parte del grupo de testcases a
evaluar. En los casos en que se requiere, se precisan observaciones entre los
resultados de evaluación esperados para la técnica de análisis utilizada por
FlowDroid y la técnica de análisis propuesta en el presente trabajo. En
la sección \ref{sec:testcases} de los anexos, se encuentra la
descripción del grupo de prueba completo.

El conjunto de prueba es analizado con FlowDroid, JoDroid\footnote{En el caso de
JoDroid, algunas son ejecutadas otras probadas **} y con el Prototipo. Los
resultados del análisis que devuelve cada herramienta son calificados como:
Falso Positivo(FP) cuando se detecta un leak que no existe; Falso Negativo(FN)
cuando no se detecta un leak existente; Verdadero Positivo(TP) cuando se detecta
un leak existente; Verdadero Negativo(TN) cuando no existe leak que detectar.

En base a estos resultados se calcula la Precisión y el Recall. La
Precisión(\textit{p}) mide si la herramienta detecta fugas cuando efectivamente
las hay(no reporta falsos positivos TP), el Recall(\textit{r}) mide si la
herramienta deja pasar fugas de información( no reporta falsos negativos FN).
Las formulas \ref{p} y \ref{r}, permiten el calculo de ambas métricas de
seguridad.

\begin{table}[H]
\small\addtolength{\tabcolsep}{-3pt}
\caption{Descripción aplicaciones de prueba.\newline
Se considera con nivel de seguridad alto, variables y métodos que almacenan y
modifican(respectivamente), información catalogada como privada(Sources).\newline 
Se considera con nivel de seguridad bajo, canales para envío de mensajes,
muestra de logs y canales creados durante el control de flujo del programa.\newline }
\label{tab:descripApps0}
\begin{tabular}{|p{13cm}|p{1cm}|}
	\hline
	\multicolumn{2}{|>{\columncolor[gray]{0.8}}c|}{\textbf{AndroidSpecific\_DirectLeak1}}\\
	\hline
	\textbf{Descripción} & \textbf{Leaks}\\
	\hline
	La variable \textit{mrg} tiene un nivel de seguridad alto,
	almacena información retornada por el método source \textit{getDeviceId}. Se
	genera flujo de información directo entre información con nivel de seguridad alto e
	información con nivel de seguridad bajo, al enviar como parámetro del método
	\textit{sendTextMessage}, información de la variable \textit{mrg}. & 1 \\
	\hline
	\multicolumn{2}{|>{\columncolor[gray]{0.8}}c|}{\textbf{AndroidSpecific\_LogNoLeak}}\\
	\hline
	\textbf{Descripción} & \textbf{Leaks}\\
	\hline
	El caso de prueba no presenta información con niveles de seguridad alto. Se
	presentan flujos de información entre información con el mismo nivel de
	seguridad, en este caso bajo, lo cual es permitido. & 0 \\
	\hline
	\multicolumn{2}{|>{\columncolor[gray]{0.8}}c|}{\textbf{ArraysAndLists\_ArrayAccess1}}\\
	\hline
	\textbf{Descripción} & \textbf{Leaks}\\
	\hline
	Se tiene un array en que se almacena información tanto proveniente como no
	proveniente de sources, parte de la información que almacena es enviada como
	parámetro del método \textit{sendTextMessage}. \textit{Observación:}
	Para la técnica de análisis de FlowDroid(taint analysis), se marca únicamente el
	índice del array donde se almacena el dato considerado como source, así,
	cuando se envía como parámetro del método \textit{sendTextMessage},
	el dato de un índice no marcado, no se genera leak. Para nuestra técnica
	de análisis(flujo de información mediante JIF), para que un array almacene
	información con nivel de seguridad alto, primero debe ser catalogo(anotado)
	con nivel de seguridad alto, lo que implica que el array podrá almacenar
	información tanto de nivel de seguridad alto como bajo, pero toda la
	información quedará con nivel de seguridad alto. En consecuencia, al enviar
	cualquier índice del array como parámetro del método 
	\textit{sendTextMessage} se presenta un flujo de información no
	permitido. & 0
	\\
	\hline
	\multicolumn{2}{|>{\columncolor[gray]{0.8}}c|}{\textbf{GeneralJava\_Exceptions2}}\\
	\hline
	\textbf{Descripción} & \textbf{Leaks}\\
	\hline
	La variable \textit{imei} es de nivel de seguridad alto, almacena información
	devuelta por el método \textit{getDeviceId}. El control de flujo del
	programa conduce de manera implícita a la captura de una excepción tipo
	RuntimeException, desde allí se utiliza información proveída por la variable
	\textit{imei}, como parámetro para invocar el método \textit{sendTextMessage}.
	Generando un flujo de información indebido. & 1
	\\
	\hline
	\multicolumn{2}{|>{\columncolor[gray]{0.8}}c|}{\textbf{ImplicitFlows\_ImplicitFlow2}}\\
	\hline
	\textbf{Descripción} & \textbf{Leaks}\\
	\hline
	 La variable \textit{userInputPassword} con nivel de seguridad alto, almacena
	 información de un campo EditText tipo textPassword(password suministrado por
	 el usuario). Se generan flujos de información indebidos: al tratar de asignar
	 información a la variable passwordCorrect con nivel de seguridad bajo, a
	 partir de la comparación de información con nivel de seguridad alto(variable
	 textPassword), después, al tratar de mostrar en el \textit{log} información
	 que depende de tal comparación. & 1\\
	\hline
\end{tabular}
\end{table}

\subsection{Comparación evaluación FlowDroid y Prototipo } 
\label{subsec:fvsp}
son presentados en la tabla\ref{tb:resultados}. En esta,
por cada caso de prueba se indica la cantidad de leaks que presenta, el
resultado y el tiempo que tarda el análisis para ambas herramientas.
El tiempo que tarda cada herramienta para evaluar el respectivo testcase es
medido con el comando \textit{time}.

\begin{table}[H]
\begin{center}
\small\addtolength{\tabcolsep}{-3pt}
\caption{Resultados de evaluación para FlowDroid y Prototipo. Donde
\textit{Testcase} especifica el nombre de la aplicación que se está evaluando;
\textit{Leaks} indica si el testcase presenta fugas de información;
\textit{F} y  \textit{P} muestran los resultados devueltos por FlowDroid y por
el Prototipo; \textit{tF} y \textit{tP}, señalan el tiempo(en segundos) que toma
el análisis para Flowdroid y para el Prototipo, respectivamente.}
\label{tb:resultados}
\begin{tabular}{|p{6cm}|p{1cm}|p{1cm}|p{1cm}|p{1cm}|p{1cm}|}
	\hline
	\textbf{Testcase} & \textbf{Leaks} & \textbf{F} &
	\textbf{P} & \textbf{ tF} & 
	\textbf{tP}\\
	\hline
	AndroidSpecific\_DirectLeak1 & 1 & TP & TP &5.371s &2.063s\\
	\hline
	AndroidSpecific\_InactiveActivity & 0 & TN & FP  &3.255s &2.469s\\
	\hline
	AndroidSpecific\_LogNoLeak & 0 & TN & TN &5.505s &2.946s\\
	\hline
	AndroidSpecific\_Obfuscation1 & 1 & TP & TP &6.734s &2.706s\\
	\hline
	 AndroidSpecific\_PrivateDataLeak2 & 1 & TP & TP & 6.144s &2.644s\\
	\hline
	 ArraysAndLists\_ArrayAccess1 & 0 & FP & FP & 4.708s & 1.278s\\
	\hline
	 ArraysAndLists\_ArrayAccess2 & 0 & FP & FP & 4.4s &1.361s\\
	 \hline
	 GeneralJava\_Exceptions1 & 1 & TP & TP &6.397s &2.755s\\
	\hline
	 GeneralJava\_Exceptions2 & 1 & TP & TP &5.887s &1.980s\\
	\hline
	GeneralJava\_Exceptions3 & 0 & FP & FP &6.008s &2.032s\\
	\hline
	GeneralJava\_Exceptions4 & 1 & TP & TP &5.731s &2.313s\\
	\hline
	GeneralJava\_Loop1 & 1 & TP & TP &5.605s &2.800s\\
	\hline
	GeneralJava\_Loop2 & 1 & TP & TP &4.719s &1.361s\\
	\hline
	GeneralJava\_UnreachableCode & 0 & TN & FP &3.792s &1.197s\\
	\hline
	ImplicitFlows\_ImplicitFlow1 & 1 & FN & TP &4.853s &1.331s\\
	\hline
	ImplicitFlows\_ImplicitFlow2 & 1 & FN & TP &4.496s &1.212s\\
	\hline
	ImplicitFlows\_ImplicitFlow4 & 1 & FN & TP &4.375s &1.224s\\
	\hline
	Lifecycle\_ActivityLifecycle3 & 1 & TP & TP &4.792s &1.222s\\
	\hline
	Lifecycle\_BroadcastReceiverLifecycle1 & 1 & TP & TP &4.456s &1.061s\\
	\hline
	Lifecycle\_ServiceLifecycle1 & 1 & TP & TP &5.225s &1.180s\\
	\hline
\end{tabular}
\end{center}
\end{table}
\subsubsection{Analisis de evaluación entre FlowDroid y Prototipo}
-\textit{Resultados de precisión}\newline
En lo que respecta a los resultados del Prototipo, los FP correspondientes a
AndroidSpecific\_InactiveActivity y GeneralJava\_UnreachableCode, surgen como
consecuencia de realizar el análisis asumiendo que el desarrollador utiliza lo
que implementa.\newline 
Por otro lado, en el caso de ArraysAndLists\_ArrayAccess1 y
ArraysAndLists\_ArrayAccess2, no es sencillo calificar los resultados como FP,
puesto que, para lo que está analizando FlowDroid(verificar que su técnica de
análisis diferencie entre los elementos marcados y no marcados de un array),
efectivamente se presentan FP, sin embargo, para la forma en que se deben
implementar los programas en jif, donde se suele definir un nivel de seguridad
para todo el array antes de almacenar los elementos en el mismo, podría decirse
que no se trata de un FP, porque se revelo información que había sido
catalogada con nivel de seguridad alto.\newline 
La detección de flujos implícitos podría ser un elemento diferenciador.

-\textit{Resultados de desempeño}\newline
Se podría destacar como positivo que el análisis de flujo de información
mediante técnicas de tipado de seguridad, requiere menos tiempo que la técnica
de marcado de datos utilizada por FlowDroid.

-\textit{Acerca de por qué FlowDroid no detecta flujos implícitos}\newline
El análisis de FlowDroid utiliza técnicas DataFlow, específicamente, utiliza
tainting análisis. Para hacer seguimiento al flujo de información de un
programa, la técnica de análisis tainting se basa en: asociar una o más marcas
con el valor de los datos en el programa, y en propagarlas. Dependiendo de los
criterios definidos para el análisis, la marca puede ser propagada a causa de
flujos explícitos o de flujos implícitos, o a causa de ambos. En flujos
explícitos la propagación ocurre cuando el valor de una variable marcada está
implicada en el calculo de otra variable. En flujos implícitos la propagación
tiene lugar a través de dependencias en el control de flujo del programa, por
ejemplo, cuando el valor de un dato marcado afecta indirectamente otra variable.\newline 
En el caso de FlowDroid, los criterios que fundamentan el análisis de la
herramienta, hacen que el marcado de datos se propague para flujos explícitos y
y no para flujos implícitos. Por consiguiente, FlowDroid no detecta flujos
implícitos.

%En el caso de TaintDroid, tampoco detecta flujos implícitos, porque entre las
% desiciones de diseño de la herramienta está enfocarse en el seguimeinto al
% flujo de datos y no al flujo de control, puesto que si incluyen seguimiento al
% flujo de control, se adiciona sobrecarga a la herramienta, la cual es de tipo
% dinámico.

\begin{table}[H]
\begin{center}
\caption{Comparación de precisión entre FlowDroid y Prototipo.\newline
Resume el total de respuestas devuelta por cada herramienta, para cada uno de
los cuatro tipos de calificación(FP, TP, TN, FN)}
\label{tb:precision}
\begin{tabular}{cc|c|c}
\cline{2-3}
& \multicolumn{0}{ |c|  }{\multirow{1}{*}{FlowDroid} } & Prototipo \\
\cline{1-3}
\multicolumn{0}{ |c|  }{\multirow{0}{*}{FP} }  & 3 & 5 &  \\ \cline{0-2}
\multicolumn{0}{ |c|  }{\multirow{0}{*}{FN} }  & 3 & 0 &  \\ \cline{0-2}
\multicolumn{0}{ |c|  }{\multirow{0}{*}{TP} }  & 11 & 14 &  \\ \cline{0-2}
\multicolumn{0}{ |c|  }{\multirow{0}{*}{TN} }  & 3 & 1 &  \\ \cline{0-2}
\end{tabular}
\end{center}
\end{table}

El conjunto de prueba está conformado por 20 aplicaciones, de las cuales 14
presentan leaks.
Los resultados obtenidos en la tabla \ref{tb:precision} señalan que de las 14
fugas existentes, el Prototipo las detecta todas, presenta 14 TP(verdaderos
positivos); mientras que, FlowDroid deja pasar 3.\newline
Por otro lado, el Prototipo presenta más falsos positivos que FlowDroid, de los
6 testcases que no presentan leaks, el prototipo identifica 5 como si fuesen
fugas, mientras que FlowDroid identifica 3.\newline
A partir de tales resultados se mide la Precisión y Recall para ambas
herramientas. La precisión \textit{p} mide si la herramienta detecta fugas
cuando efectivamente las hay(no reporta falsos positivos TP), el Recall
\textit{r} mide si la herramienta deja pasar fugas de información( no reporta
falsos negativos FN).
Las formulas para ambas métricas son:
\begin{lstlisting}
Precision p = TP/(TP +FP)
Recall r = TP/(TP+FN) 
\end{lstlisting} 
Donde TP representa el total de verdaderos positivos, FP el
total de falsos positivos y  FN el total de falsos negativos.\newline

En lo que respecta a precisión, FlowDroid presenta un porcentaje del 78\%,
siendo más preciso frente al Prototipo, que presenta un porcentaje del
73\%.\newline 
Por otro lado, el Prototipo presenta un porcentaje en Recall del 100\%,
mientras que FlowDroid presenta un porcentaje del 78\%.

\subsection{Comparación evaluación JoDroid y Prototipo}
\label{subsec:jvsp}
\begin{table}[H]
\begin{center}
\small\addtolength{\tabcolsep}{-3pt}
\caption{Resultados de evaluación para JoDroid y Prototipo. Donde
\textit{Testcase} especifica el nombre de la aplicación que se está evaluando;
\textit{Leaks} indica si el testcase presenta fugas de información; \textit{J} y
\textit{P} muestran los resultados devueltos por JoDroid y por el Prototipo;
\textit{tJ} y \textit{tP}, señalan el tiempo que toma el análisis para JoDroid
y para el Prototipo, respectivamente.}
\label{tb:resultados}
\begin{tabular}{|p{6cm}|p{1cm}|p{1cm}|p{1cm}|p{1cm}|p{1cm}|}
	\hline
	\textbf{Testcase} & \textbf{Leaks} & \textbf{J} &
	\textbf{P} & \textbf{ tJ} & 
	\textbf{tP}\\
	\hline
	AndroidSpecific\_DirectLeak1 & 1 & TP & TP &&2.063s\\
	\hline
	AndroidSpecific\_InactiveActivity & 0 & FP & FP  &&2.469s\\
	\hline
	AndroidSpecific\_LogNoLeak & 0 & TN & TN &&2.946s\\
	\hline
	AndroidSpecific\_Obfuscation1 & 1 & TP & TP &&2.706s\\
	\hline
	 AndroidSpecific\_PrivateDataLeak2 & 1 & TP & TP &&2.644s\\
	\hline
	 ArraysAndLists\_ArrayAccess1 & 0 & FP & FP && 1.278s\\
	\hline
	 ArraysAndLists\_ArrayAccess2 & 0 & FP & FP &&1.361s\\
	 \hline
	 GeneralJava\_Exceptions1 & 1 & - & TP &&2.755s\\
	\hline
	 GeneralJava\_Exceptions2 & 1 & - & TP &&1.980s\\
	\hline
	GeneralJava\_Exceptions3 & 0 &  - & FP &&2.032s\\
	\hline
	GeneralJava\_Exceptions4 & 1 & - & TP &&2.313s\\
	\hline
	GeneralJava\_Loop1 & 1 & TP & TP &&2.800s\\
	\hline
	GeneralJava\_Loop2 & 1 & TP & TP &&1.361s\\
	\hline
	GeneralJava\_UnreachableCode & 0 & TN & FP &&1.197s\\
	\hline
	ImplicitFlows\_ImplicitFlow1 & 1 & TP & TP &&1.331s\\
	\hline
	ImplicitFlows\_ImplicitFlow2 & 1 & TP & TP &&1.212s\\
	\hline
	ImplicitFlows\_ImplicitFlow4 & 1 & TP & TP &&1.224s\\
	\hline
	Lifecycle\_ActivityLifecycle3 & 1 & TP & TP &&1.222s\\
	\hline
	Lifecycle\_BroadcastReceiverLifecycle1 & 1 & TP & TP &&1.061s\\
	\hline
	Lifecycle\_ServiceLifecycle1 & 1 & TP & TP &&1.180s\\
	\hline
\end{tabular}
\end{center}
\end{table}

\subsubsection{Analisis de evaluación entre JoDroid y Prototipo}
JoDroid no evalua los testcases para excepciones, puesto que la actual versión
no soporta en analisis de flujo de información para excepciones.
\cite{JoDroid-Thesis} Pag 93\newline

\begin{table}[H]
\begin{center}
\caption{Comparación de precisión entre JoDroid y Prototipo.\newline
Resume el total de respuestas devuelta por cada herramienta, para cada uno de
los cuatro tipos de calificación(FP, TP, TN, FN)}
\label{tb:precision}
\begin{tabular}{cc|c|c}
\cline{2-3}
& \multicolumn{0}{ |c|  }{\multirow{1}{*}{JoDroid} } & Prototipo \\
\cline{1-3}
\multicolumn{0}{ |c|  }{\multirow{0}{*}{FP} }  & 3 & 4 &  \\ \cline{0-2}
\multicolumn{0}{ |c|  }{\multirow{0}{*}{FN} }  & 0 & 0 &  \\ \cline{0-2}
\multicolumn{0}{ |c|  }{\multirow{0}{*}{TP} }  & 11 & 11 &  \\ \cline{0-2}
\multicolumn{0}{ |c|  }{\multirow{0}{*}{TN} }  & 2 & 1 &  \\ \cline{0-2}
\end{tabular}
\end{center}
\end{table}

En lo que respecta a la métrica de Precisión, JoDroid presenta un
porcentaje del 78,57\%; frente al Prototipo que presenta un porcentaje
del 73,33\%.\newline
Para la métrica de Recall, tanto JoiDroid como el Prototipo, presentan
el mismo porcentaje esto es 100\%.\newline
\subsection{Comparación FlowDroid, JoDroid, Prototipo}
\label{subsec:fjp}


-\textit{Qué tanto cambia la anotación del código original}
\textbf{Martín: está parte debería ir el esta sección, o en la discusión?}

























