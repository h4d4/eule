\label{ch:evaluacion}
\chapter{Evaluación}
Ventajas y limitaciones de la solución.\newline 
Si aplica, evaluación de desempeño.  \newline 
Si aplica, evaluación de usabilidad.  
Hay otras soluciones similares? \newline 
Cuáles son las diferencias y las ventajas y desventajas con respecto a esas soluciones.

\section{Consideraciones de evaluación}
No se consideran flujos de información vía interAppComunicación. Por ejemplo,
varias aplicaciones que se comunican entre sí.

\section{Evaluación conjunto de aplicaciones}
Para la evaluación del prototipo se toman un grupo de testcases de DroidBech, el
benchmark de FlowDroid, aplicables para la evaluación de la política de
seguridad establecida.\newline 
Se considera con nivel de seguridad alto, variables y métodos que almacenan y
modifican(respectivamente), información considereada como privada(Sources).\newline 
Se considera con nivel de seguridad bajo, canales para envío de mensajes,
muestra de logs y canales creados durante el flujo del programa.\newline
A continuación se describen los testcases a evaluar, en los casos en que se
requiere, se precisan observaciones entre los resultados de evaluación esperados
para la técnica de análisis utilizada por FlowDroid y la técnica de análisis
% propuesta en el presente trabajo.\newline 
% \textbf{AndroidSpecific\_DirectLeak1}\newline
% La variable \textit{mrg} tiene un nivel de seguridad alto, almacena
% información retornada por el método source \textit{getDeviceId}. Se genera
% flujo de información directo entre información con nivel de seguridad alto e
% información con nivel de seguridad bajo, al enviar como parámetro del método
% \textit{\textbf{sendTextMessage}}, información de la variable \textit{\textbf{mrg}}.

\begin{table}[H]
\small\addtolength{\tabcolsep}{-3pt}
\caption{Descripción aplicaciones de prueba}
\label{tb:descripApps}
\begin{tabular}{|p{13cm}|p{1cm}|}
	\hline
	\multicolumn{2}{|>{\columncolor[gray]{0.8}}c|}{\textbf{AndroidSpecific\_DirectLeak1}}\\
	\hline
	\textbf{Descripción} & \textbf{Leaks}\\
	\hline
	La variable \textit{mrg} tiene un nivel de seguridad alto,
	almacena información retornada por el método source \textit{getDeviceId}. Se
	genera flujo de información directo entre información con nivel de seguridad alto e
	información con nivel de seguridad bajo, al enviar como parámetro del método
	\textit{sendTextMessage}, información de la variable \textit{mrg}. & 1 \\
	\hline
	\multicolumn{2}{|>{\columncolor[gray]{0.8}}c|}{\textbf{AndroidSpecific\_InactiveActivity}}\\
	\hline
	\textbf{Descripción} & \textbf{Leaks}\\
	\hline 
	La variable \textit{imei} tiene un nivel de seguridad alto, almacena
	información retornada por el source getDeviceId. La variable es enviada como
	parametro a \textit{Log}, canal que muestra información con nivel de
	seguridad bajo. \textit{Observación:} debido a que la actividad en que se
	presenta este flujo de información no está activada en el Manifest de la
	aplicación, para la técnica de análisis de FlowDroid no existen leaks. Para
	nuestra propuesta de análisis si existe leak, porque se asume que los métodos y
	sus aplicaciones podrán ser ejecutados. & 0
	\\
	\hline
	\multicolumn{2}{|>{\columncolor[gray]{0.8}}c|}{\textbf{AndroidSpecific\_LogNoLeak}}\\
	\hline
	\textbf{Descripción} & \textbf{Leaks}\\
	\hline
	El caso de prueba no presenta información con niveles de seguridad alto. Se
	presentan flujos de información entre información con el mismo nivel de
	seguridad, en este caso bajo, lo cual es permitido. & 0 \\
	\hline
	\multicolumn{2}{|>{\columncolor[gray]{0.8}}c|}{\textbf{AndroidSpecific\_Obfuscation1}}\\
	\hline
	\textbf{Descripción} & \textbf{Leaks}\\
	\hline 
	La variable \textit{\textbf{mrg}} tiene un nivel de seguridad alto,
	almacena información retornada por el método source getDeviceId().
	Se genera flujo de información entre información con nivel de seguridad alto e
	información con nivel de seguridad bajo, al enviar como parámetro del método
	\textit{sendTextMessage}, información de la variable
	\textit{mrg}. \textit{Observación:} el elemento adicional para este
	testcase es proveer una suplantación de la clase
	android.telephony.TelephonyManager, en el apk de la aplicación. Para la
	evaluación que proponemos, se verifica acorde a la versión que se tiene anotada
	para esta clase, es decir, independientemente de la obfuscación de la clase,
	nuestro análisis debe detectar que existe un flujo de información indebido. &
	1\\
	\hline
	\multicolumn{2}{|>{\columncolor[gray]{0.8}}c|}{\textbf{AndroidSpecific\_PrivateDataLeak2}}\\
	\hline
	\textbf{Descripción} & \textbf{Leaks}\\
	\hline
	La variable \textit{info} tiene un nivel de seguridad alto, almacena
	información suministrada por el campo EditText de tipo textPassword. Se genera
	flujo de información entre información con nivel de seguridad alto e
	información con nivel de seguridad bajo, al pasar la variable
	\textit{info} como parámetro de \textit{Log}, que muestra
	información con nivel de seguridad bajo. & 1 
	\\
	\hline
	\multicolumn{2}{|>{\columncolor[gray]{0.8}}c|}{\textbf{ArraysAndLists\_ArrayAccess1}}\\
	\hline
	\textbf{Descripción} & \textbf{Leaks}\\
	\hline
	Se tiene un array en que se almacena información tanto proveniente como no
	proveniente de sources, parte de la información que almacena es envíada como
	parámetro del método \textit{sendTextMessage}. \textit{Observación:}
	Para la técnica de análisis de FlowDroid(tainting), se marca únicamente el
	indice del array donde se almacena el dato considerado como source, así,
	cuando se envía como parámetro del método \textit{sendTextMessage},
	el dato de un indice no marcado, no se genera leak. Para nuestra técnica
	de análisis(flujo de información mediante JIF), para que un array almacene
	información con nivel de seguridad alto, primero debe ser catalogo(anotado)
	con nivel de seguridad alto, lo que implica que el array podrá almacenar
	información tanto de nivel de seguridad alto como bajo, pero toda la
	información quedará con nivel de seguridad alto. En consecuencia, al enviar
	cualquier indice del array como parámetro del metódo 
	\textit{sendTextMessage} se presenta un flujo de información no
	permitido, se genera un leak. & 0
	\\
	\hline
\end{tabular}
\end{table}

\begin{table}[H]
\small\addtolength{\tabcolsep}{-3pt}
\caption{Descripción aplicaciones de prueba}
\label{tb:descripApps}
\begin{tabular}{|p{13cm}|p{1cm}|}
	\hline
	\multicolumn{2}{|>{\columncolor[gray]{0.8}}c|}{\textbf{ArraysAndLists\_ArrayAccess2}}\\
	\hline
	\textbf{Descripción} & \textbf{Leaks}\\
	\hline
	Se presenta el contexto descrito en ArraysAndLists\_ArrayAccess1, con un
	elemento adicional, se implementa el método calculateIndex(), que calcula el
	indice del array a ser envíado como parámetro del método
	\textit{sendTextMessage}. & 0 \\
	\hline
	\multicolumn{2}{|>{\columncolor[gray]{0.8}}c|}{\textbf{GeneralJava\_Exceptions1}}\\
	\hline
	\textbf{Descripción} & \textbf{Leaks}\\
	\hline
	La variable \textit{imei} es de nivel de seguridad alto, almacena información
	devuelta por el método \textit{getDeviceId}. Se genera flujo de información
	entre información de nivel de seguridad alto e información con nivel de
	seguridad bajo, al enviar como parametro del método \textit{sendTextMessage}
	información de la variable \textit{imei}. Este flujo de información se presenta
	dentro de la captura de una excepción RuntimeException(no es verificada
	en tiempo de compilación).
	& 1
	\\
	\hline
	\multicolumn{2}{|>{\columncolor[gray]{0.8}}c|}{\textbf{GeneralJava\_Exceptions2}}\\
	\hline
	\textbf{Descripción} & \textbf{Leaks}\\
	\hline
	La variable \textit{imei} es de nivel de seguridad alto, almacena información
	devuelta por el método \textit{getDeviceId}. El control de flujo del
	programa conduce de manera implícita a la captura de una excepción tipo
	RuntimeException, desde allí se utiliza informacoión proveida por la variable
	\textit{imei} como parámetro para invocar el método \textit{sendTextMessage}.
	Generando un flujo de información indebido. & 1
	\\
	\hline
	\multicolumn{2}{|>{\columncolor[gray]{0.8}}c|}{\textbf{GeneralJava\_Exceptions3}}\\
	\hline
	\textbf{Descripción} & \textbf{Leaks}\\
	\hline
	La variable \textit{imei} es de nivel de seguridad alto, almacena información
	devuelta por el método \textit{getDeviceId}. La información proveida por
	\textit{imei} es utilizada como parámetro para invocar el método
	\textit{sendTextMessage} dentro de la captura de una excepción tipo
	RuntimeException, sin embargo, el programa no genera un caso que haga ejecutar
	la captura de la excepción. & 0
	\\
	\hline
	\multicolumn{2}{|>{\columncolor[gray]{0.8}}c|}{\textbf{GeneralJava\_Exceptions4}}\\
	\hline
	\textbf{Descripción} & \textbf{Leaks}\\
	\hline
	La variable \textit{imei} es de nivel de seguridad alto, almacena información
	devuelta por el método \textit{getDeviceId}. información proveída por esta
	variable es envíada como parámetro para la captura de una excepción en tiempo
	de ejecucón, donde es utilizado como parámetro para invocar el método
	\textit{sendTextMessage}, generando un flujo de información indebido. & 1\\
	\hline
	\multicolumn{2}{|>{\columncolor[gray]{0.8}}c|}{\textbf{GeneralJava\_Loop1}}\\
	\hline
	\textbf{Descripción} & \textbf{Leaks}\\
	\hline
	La variable \textit{imei} es de nivel de seguridad alto, almacena información
	devuelta por el método \textit{getDeviceId}. Se generan flujos de información
	indebidos, primero al tratar de asignar la información de la varianle a un
	array con nivel de seguridad bajo(donde se intenta ofuscar la información),
	luego al tratar de enviar la información ofuscada como paramétro del método
	\textit{sendTextMessage}, con nivel de seguridad bajo. & 1 \\
	\hline
	\multicolumn{2}{|>{\columncolor[gray]{0.8}}c|}{\textbf{GeneralJava\_Loop2}}\\
	\hline
	\textbf{Descripción} & \textbf{Leaks}\\
	\hline
	La variable \textit{imei} es de nivel de seguridad alto, almacena información
	devuelta por el método \textit{getDeviceId}. Se busca ofuscar la información de
	\textit{imei} mediante ciclos for anidados, allí se asigna la información de la
	varianle a un array con nivel de seguridad bajo. Luego se envía la información
	ofuscada, como parámetro del método \textit{sendTextMessage}, con nivel de
	seguridad bajo, gerendo otro flujo de información indebido. & 1\\
	\hline
\end{tabular}
\end{table}

\begin{table}[H]
\small\addtolength{\tabcolsep}{-3pt}
\caption{Descripción aplicaciones de prueba}
\label{tb:descripApps}
\begin{tabular}{|p{13cm}|p{1cm}|}
	\multicolumn{2}{|>{\columncolor[gray]{0.8}}c|}{\textbf{GeneralJava\_UnreachableCode}}\\
	\hline
	\textbf{Descripción} & \textbf{Leaks}\\
	\hline
	La variable \textit{deviceid} con nivel de seguridad alto, está contenida en un
	método que no es llamado, dentro del mismo, \textit{deviceid} es pasada como
	parámetro para invocar el método \textit{sendTextMessage}, con nivel de
	seguridad bajo. \textit{Observaciones:} para el análisis de FlowDroid el
	programa no presenta leaks, ya que el método nunca es llamado.
	Para nuestro análisis, el programa presenta leak porque se asume que todos los
	métodos son llamados. & 0\\
	\hline
	\multicolumn{2}{|>{\columncolor[gray]{0.8}}c|}{\textbf{ImplicitFlows\_ImplicitFlow1}}\\
	\hline
	\textbf{Descripción} & \textbf{Leaks}\\
	\hline
	 La variable \textit{imei} almacena información devuelta por el método,
	 \textit{getDeviceId}, con nivel de seguridad alto, \textit{imei} es pasada
	 como parámetro al método obfuscateIMEI que devuelve la información ofuscada.
	 Después se invoca el método WriteToLog, con la información ofuscada como
	 parámetro para ser mostrada en el log. Al invocar el método WriteToLog con la
	 información ofuscada, se genera un flujo de información indebido. & 1 \\
	\hline
	\multicolumn{2}{|>{\columncolor[gray]{0.8}}c|}{\textbf{ImplicitFlows\_ImplicitFlow2}}\\
	\hline
	\textbf{Descripción} & \textbf{Leaks}\\
	\hline
	 La variable \textit{userInputPassword} con nivel de segurida alto, almacena
	 información de un campo EditText tipo textPassword. Se generan flujos de
	 información indebidos: al tratar de asignar información a la variable
	 passwordCorrect con nivel de seguridad bajo, a partir de la comparación de
	 información con nivel de seguridad alto(variable textPassword), después, al
	 tratar de mostrar en el \textit{log} información que depende de tal
	 comparación. & 1
	 \\
	\hline
	\multicolumn{2}{|>{\columncolor[gray]{0.8}}c|}{\textbf{ImplicitFlows\_ImplicitFlow4}}\\
	\hline
	\textbf{Descripción} & \textbf{Leaks}\\
	\hline
	 & \\
	\hline
\end{tabular}
\end{table}

\begin{table}[H]
\small\addtolength{\tabcolsep}{-3pt}
\caption{Comparación Precisión entre FlowDroid y Prototipo}
\label{tb:comparacion}
\begin{tabular}{|p{5.8cm}|p{1cm}|p{2.1cm}|p{2.1cm}|p{1cm}|p{1cm}|}
	\hline
	\textbf{Testcase} & \textbf{Leaks} & \textbf{FlowDroid} &
	\textbf{Prototipo} & \textbf{ t F} & 
	\textbf{t P}\\
	\hline
	AndroidSpecific\_DirectLeak1 & 1 & detecta leak & detecta leak &5.371s &2.063s\\
	\hline
	AndroidSpecific\_InactiveActivity & 0 & cero leaks & FP  &3.255s &2.469s\\
	\hline
	AndroidSpecific\_LogNoLeak & 0 & cero leaks & cero leaks &5.505s &2.946s\\
	\hline
	 AndroidSpecific\_Obfuscation1 & 1 & detecta leak & detecta leak &6.734s
	 &2.706s\\
	\hline
	 AndroidSpecific\_PrivateDataLeak2 & 1 & detecta leak & detecta leak &
	 6.144s &2.644s\\
	\hline
	 ArraysAndLists\_ArrayAccess1 & 0 & & & &\\
	\hline
	 ArraysAndLists\_ArrayAccess2 & 0 & & & &\\
	 \hline
	 GeneralJava\_Exceptions1 & 1 & detecta leak & detecta leak &6.397s &2.755s\\
	\hline
	 GeneralJava\_Exceptions2 & 1 & detecta leak & detecta leak &5.887s &1.980s\\
	\hline
	GeneralJava\_Exceptions3 & 0 & FP & FP &6.008s &2.032s\\
	\hline
	GeneralJava\_Exceptions4 & 1 & detecta leak & detecta leak &5.731s &2.313s\\
	\hline
	GeneralJava\_Loop1 & 1 & detecta leak & detecta leak\footnote{} &5.605s
	&2.800s\\
	\hline
	GeneralJava\_Loop2 & 1 & detecta leak & detecta leak\footnote{} &4.719s
	&1.361s\\
	\hline
	GeneralJava\_UnreachableCode & 1 & cero leaks & FP &3.792s &1.197s\\
	\hline
	ImplicitFlows\_ImplicitFlow1 & 1 & FN & detecta leak &4.853s &1.331s\\
	\hline
	ImplicitFlows\_ImplicitFlow2 & 1 & FN & detecta leak &4.496s &1.212s\\
	\hline
	ImplicitFlows\_ImplicitFlow4 & 1 & FN & detecta leak &4.375s &1.224s\\
	\hline
	Lifecycle\_ActivityLifecycle3 & 1 & detecta leak & detecta leak &4.792s
	&1.222s\\
	\hline
	Lifecycle\_BroadcastReceiverLifecycle1 & 1 & detecta leak & detecta leak
	&4.456s &1.061s\\
	\hline
	Lifecycle\_ServiceLifecycle1 & 1 & detecta leak &detecta leak &5.225ss
	&1.180s\\
	\hline
	& 0 & & & &\\
	\hline
\end{tabular}
\end{table}
