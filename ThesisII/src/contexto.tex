\section{Contexto}
\label{sec:contexto}
%-(Lina: falta actualizar contenido)\newline
Las soluciones propuestas para detectar fuga de información en aplicaciones
Android, se enmarcan en el análisis estático o dinámico de la aplicación, en
algunos casos, se combinan ambos tipos.\newline 
En análisis estático, se estudia el código del programa para inferir todos
los posibles caminos de ejecución. Esto se logra construyendo modelos de estado
del programa, y determinando los estados posibles que puede alcanzar el
programa.
No obstante, debido a que existen múltiples posibilidades de ejecución, se opta
por construir un modelo abstracto de los estados del programa. La consecuencia
de tener un modelo aproximado es pérdida de información y posibilidad de menor
precisión en el análisis.\newline 
Por otro lado, en análisis dinámico se ejecuta el programa y se analiza su
comportamiento, verificando el camino de ejecución que ha tomado el programa.
Esa exactitud en la ejecución que se verifica, da precisión al análisis, porque
no es necesario construir un modelo aproximado de todos los posibles caminos de
ejecución.\newline 
Aunque los resultados del análisis estático pueden perder precisión, la ventaja
es que son generalizables, es decir, el modelo construido representa una
descripción del comportamiento del programa, independientemente de las entradas
y el contexto en que este se ejecute. Ahora, con el análisis dinámico, no es
posible generalizar sus resultados para futuras ejecuciones, porque no
existen garantías de que las entradas para las cuales fue ejecutado el programa,
contengan características para todos los posibles caminos de ejecución.\newline 
Además de las ventajas y desventajas de ambas clases de análisis, cada uno
implica su propio reto, así, mientras en el análisis estático la dificultad está
en construir el modelo de abstracción adecuado, en el análisis dinámico, es
complejo encontrar un conjunto de casos de prueba representativo, a analizar
durante la ejecución del programa.\newline
Por otra parte, dependiendo de la finalidad con qué se detecte la fuga de
información, un tipo de análisis puede ser más apropiado que otro. Si se busca
contener la fuga de información a tiempo de ejecución, análisis dinámico es el
camino apropiado. Por el contrario, si se busca garantizar que a tiempo de
ejecución la aplicación no incurrirá en fugas de información, resulta más
conveniente aplicar análisis estático, porque cumplir con tales garantías
implica definir políticas de confidencialidad y/o integridad desde la
implementación de la aplicación.\newline 
Precisamente, el propósito fundamental del presente trabajo es ofrecer al
desarrollador de aplicaciones Android una herramienta para aplicar políticas de
confidencialidad en la aplicación que implementa, así, la aplicación se
ejecutará exitosamente, si y sólo si, cumple con las políticas definidas, de lo
contrario, el desarrollador puede revisar y corregir su código.\newline
 
En análisis estático, generalmente, se aplican técnicas de seguridad de
tipado(Typed-Inference/Security-Typed Analysis) y técnicas de flujo de
datos(Data/Control Flow Analysis). Con técnicas Security-Typed las propiedades
de confidencialidad e integridad son anotadas en el código, y verificadas a
tiempo de compilación para garantizar que se cumplen en tiempo de ejecución. Con
las técnicas de flujo de control o flujo de datos, se hace seguimiento al flujo
de los datos o control de flujo para verificar el cumplimiento de políticas de
seguridad, generalmente se utilizan grafos de Control de Flujo CFG(Control Flow
Graph), Grafos de Flujo de Datos DFG( Data Flow Graph) y Grafos de llamadas CG
(Call Graphs).\newline 

Al consultar la literatura científica, es posible inferir que parte importante
de las propuestas para análisis de fuga de información en aplicativos
Android(TaintDroid\cite{TaintDroid}, Flow-Droid\cite{FlowDroid-Thesis},
DidFail\cite{DidFail}, DroidForce\cite{DroidForce}), parten del bytecode para
realizar data-flow analysis, mediante técnicas de análisis tainting. Tainting
es un tipo especial de análisis de flujo de datos, que hace seguimiento al flujo
de datos entre un conjunto de fuentes considerados privados y/o sensibles; y un
conjunto de destinos considerados no confiables, sources y sinks,
respectivamente.\newline 
Tales propuestas se enfocan en analizar aplicativos de terceros para detectar
flujos de datos indebidos, y no: para garantizar el cumplimiento de determinadas
políticas de seguridad. En consecuencia, es complejo que el desarrollador
garantice la ausencia de fugas de información en la aplicación que implementa,
partiendo de tales herramientas. Puesto que, al seguir únicamente a los datos
marcados, los datos no marcados para el análisis, pueden acarrear fugas de
información(under-tainting). Adicionalmente, si no se hace seguimiento al flujo
de control pueden existir fugas de información a través de flujos implícitos,
ya que, el análisis estará centrado en flujos explícitos.\newline
No obstante, las limitaciones propias de un análisis basado en flujo de datos,
pueden superarse enfocando el análisis de la aplicación hacia técnicas de
análisis basadas en control de flujo de información, ya que estás analizan el
aplicativo de forma estática para identificar todos los posibles caminos que
podría tomar la aplicación en tiempo de ejecución. Así, con análisis basado en
control de flujo de información, no sólo es posible prevenir fugas por
under-tainting y flujos implícitos; sino que también, es posible ofrecer
garantías del cumplimiento de determinadas políticas de seguridad.\newline 
Ahora, las reglas para evaluar control de flujo de información pueden definirse
mediante técnicas Security-Typed, por ejemplo como se definen con
JIF \ref{JIF-Tool}, una herramienta basada en lenguajes tipados de seguridad
para realizar control de flujo de información, el inconveniente es que está
implementada para aplicaciones en Java, y no para aplicativos Android.\newline 
En general, herramientas como estas, basadas en técnicas de análisis
Security-Typed, implican conceptos como flujo de información, políticas de
confidencialidad e integridad, y chequeo de tipos.\newline 
El flujo de información describe el comportamiento de un programa, desde la
entrada de los datos hasta la salida de los mismos.\newline 
Confidencialidad e integridad son políticas de seguridad, aplicables mediante
control de flujo de información. La confidencialidad busca prevenir que la
información fluya hacia destinos no apropiados, mientras que, la integridad
busca prevenir que la información provenga de fuentes no apropiadas. Una
importante diferencia entre confidencialidad e integridad, es que la integridad
de la información de un programa puede ser alterada sin la interacción con
agentes externos.\newline %\textcolor{red}{(por qué es importante?)}
Ambas políticas son fundamentales para garantizar propiedades de seguridad. Con
políticas de confidencialidad, es posible garantizar ausencia de fugas de
información. Mientras que, con políticas de integridad, la finalidad es evitar
modificación de la información, de forma no consentida.\newline
Por consiguiente, verificar que un programa utilice la información acorde a
tales políticas, implica analizar sus flujos de información, de inicio a fin.
Para el análisis, se deben definir: políticas de flujo de información y
controles de flujo de información, es decir, las políticas de seguridad a
evaluar y los mecanismos para aplicarlas.\newline 
Al usar un lenguaje tipado de seguridad, las políticas son definidas a través
del lenguaje, porque son expresadas mediante anotaciones en el código fuente del
programa a verificar, y su evaluación se realiza mediante chequeo de tipos. El
chequeo de tipos consiste en una técnica estática, también utilizada para
analizar flujo de información durante la compilación de un programa, más
específicamente en la etapa de análisis semántico, el compilador identifica el
tipo para cada expresión del programa y verifica que corresponda al contexto de
la expresión.
%  El
% chequeo de tipos también es una técnica estática utilizada para analizar flujo
% de información durante la compilación de un programa, más específicamente en la
% etapa de análisis semántico, el compilador identifica el tipo para cada
% expresión del programa y verifica que corresponda al contexto de la expresión.
Bajo este principio de chequeo, lenguajes tipados de seguridad aplican
políticas de control de flujo, definiendo para cada expresión del programa un
tipo de seguridad(security type), de la forma:  tipo de dato y label de
seguridad(security label), regulador de uso del dato, acorde a su tipo. El
compilador realiza el chequeo de tipos, partiendo del conjunto de labels de
seguridad. Así, si el programa pasa el chequeo de tipos y compila correctamente,
se espera que cumpla con las políticas de control de flujo evaluadas.
